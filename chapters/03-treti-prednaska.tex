\subsubsection{Sumarizace (agregace)}
\begin{upcode}{Počet n-tic v relaci (SQL)}{}{SQL}
SELECT COUNT(*) FROM my_table;
\end{upcode}
Výsledkem obdobného dotazu v SQL bude vždy relace. Například v tomto přípa-dě relace o jedné jednoprvkové n-tici, která obsahuje počet řádků původní relace.
\begin{upcode}{Počet n-tic v relaci (Tutorial D)}{}{TutorialD}
COUNT(relexpr)
\end{upcode}
Naopak v Tutorial D nebude výsledkem relace, avšak skalární hodnota. Tedy jednoduše číslo. Chování jazyka SQL lze v Tutorial D umitovat.
\begin{upcode}{Imitace chování SQL (Tutorial D)}{}{TutorialD}
SUMMARIZE relexpr ADD (COUNT() AS count_of_tuples)
\end{upcode}
\begin{upcode}{Pokročilá sumarizace (Tutorial D)}{code:summarized}{TutorialD}
SUMMARIZE relexpr1
PER (relexpr2)
ADD (summary AS name)

/* místo PER lze použít také BY */
/* platí, že BY {seznam atributů} odpovídá PER (relexpr1 {seznam atributů}) */
SUMMARIZE relexpr1
BY (seznam atributů)
ADD (summary AS name)
\end{upcode}
Navíc by mělo platit, že\upinlinecode{TutorialD}{!}{relexpr1} $\subseteq$\upinlinecode{TutorialD}{!}{relexpr2}. Význam zdrojového kódu \ref{code:summarized} je takový, že jdeme přes všechny n-tice z\upinlinecode{TutorialD}{!}{relexpr2} a počítáme sumarizaci ze všech n-tic z\upinlinecode{TutorialD}{!}{relexpr1}, které se shodují na atributech z relačního schématu z\upinlinecode{TutorialD}{!}{relexpr2}.

\subsubsection{Seskupování}
Uplatňuje se v modelech, které umožňují atributy s doménami, které jsou množi-ny relací.

\subsubsubsection{Metoda UNGROUP}
\begin{upcode}{Seskupování UNGROUP (Tutorial D)}{}{TutorialD}
/* atributy jsou typu relace */
relexpr UNGROUP (atribut, dalsi_atribut)
\end{upcode}
Výsledkem je relace nad schématem, které vznikne tak, že místo atributů ze seznamu budeme mít atributty vnořených tabulek a výsledek obsahuje n-tice tak, že pro každou n-tici je ve výsledku (obecně) několik n-tic, jejichž hodnoty jsou brány z tabulek ve výchozí n-tici metodou každý s každým.

\subsubsubsection{Metoda GROUP}
\begin{upcode}{Seskupování GROUP (Tutorial D)}{}{TutorialD}
/* atributy jsou typu relace */
relexpr GROUP (atribut, dalsi_atribut)
\end{upcode}
Na základě relace a seznamu atributů množin atributů, které se mají seskupit pod danými novými jmény se vytvoří relace, ve které jsou zbylé atributy výchozí relace a nové atributy, které nabývají hodnot, v nichž jsou maximální relace obsahující n.tice hodnot výchozí tabulky tak, že individuální hodnoty výsledných n-tic jsou v relaci se vzniklými relacemi právě tehdy, když zřetězení libovolných individuálních n-tic se zbylými individuálními hodnotami z každé výsledné n-tice se nachází ve výchozí tabulce.

\subsubsection{Přejmenování}
Z relace $\mathcal{D}$ nad relačním schématem $R$ se vytvoří nová relace
$$
\rho \thickspace y_{1}' \Leftarrow y_{1}, \ldots, y_{n}' \Leftarrow y_{n} (\mathcal{D})
$$
nad schématem $R$, ve kterém byly atributy $y_{1}, \ldots, y_{n}$ přejmenovány na atributy $y_{1}', \ldots, y_{n}'$, ale za předpokladu, že každé dva atributy $y_{i}, y_{i}' \text{ pro } 1 \leq i \leq n$ mají stejný typ.
\begin{upcode}{Přejmenování (Tutorial D)}{}{TutorialD}
relexpr RENAME (old AS new, old_2 AS new_2))
\end{upcode}
\begin{upcode}{Přejmenování (SQL)}{}{SQL}
SELECT old AS new, old_2 AS new_2 FROM table_1;
\end{upcode}

\subsubsection{Přirozené spojení}
\begin{uptheorem}[Přirozené spojení]
Mějme relace $\mathcal{D}_{1}$ nad schématem $R \cup S$ a $\mathcal{D}_{2}$ nad schématem $S \cup T$ tak, že $R, S, T$ jsou po dvou disjunktní a $S$ je množina všech společných atributů. Pak je přírozené spojení $\mathcal{D}_{1}, \mathcal{D}_{2}$ relace nad schématem $R \cup S \cup T$ dáno předpisem
$$
\mathcal{D}_{1} \bowtie \mathcal{D}_{2} = \lbrace rst \thickspace | \thickspace rs \in \mathcal{D}_{1}, st \in \mathcal{D}_{2} \rbrace
$$
\end{uptheorem}
\begin{upcode}{Přirozené spojení (Tutorial D)}{}{TutorialD}
/* výsledkem je TUPLE {x 10, y 20, z 30} */
TUPLE {x 10, y 20} UNION TUPLE {y 20, z 30}
\end{upcode}
\begin{upexample}[Přirozené spojení]
Přírozeným spojením získáme tabulku, který má společné záhlaví, avšak obsahuje pouze n-tice, které se shodují na společných atributech daných vstupních tabulek resp. relací. Názorně na tabulkách ve skupině tabulek \ref{tab:prir_spojeni}.
\begin{table}
\caption{Přirozené spojení tabulek}\label{tab:prir_spojeni}
\begin{subtable}[t]{0.3\textwidth}
\centering
\caption{První operand}
\begin{tabular}{c | c | c}
w & x & y \\
\hline
1 & 1 & 2 \\
2 & 2 & 2 \\
3 & 3 & 2 \\
4 & 3 & 4 \\
5 & 5 & 5
\end{tabular}
\end{subtable}
~
\begin{subtable}[t]{0.3\textwidth}
\centering
\caption{Druhý operand}
\begin{tabular}{c | c | c}
x & y & z \\
\hline
1 & 1 & 2 \\
2 & 2 & 1 \\
3 & 2 & 4 \\
4 & 4 & 5 \\
5 & 5 & 5
\end{tabular}
\end{subtable}
~
\begin{subtable}[t]{0.3\textwidth}
\centering
\caption{Výsledná tabulka}
\begin{tabular}{c | c | c | c}
w & x & y & z \\
\hline
2 & 2 & 2 & 1 \\
3 & 2 & 3 & 4 \\
5 & 5 & 5 & 5
\end{tabular}
\end{subtable}
\end{table}
\end{upexample}